\documentclass[a4paper]{article}

\usepackage[T1]{fontenc}
\usepackage[utf8]{inputenc}
\usepackage[polish]{babel}

\usepackage[left = 2cm, right = 2cm, top = 2cm, bottom = 2cm] {geometry}

\usepackage{amsmath, amsfonts}
\usepackage{textcomp}

\pagestyle{empty}

\author{}
\title{}
\date{\today}

\begin{document}
\section*{Instrukcja gry "Monopoly UWr"}

%\paragraph*{1.Informacje ogólne}
\noindent \textbf{1.Informacje ogólne}\\
\noindent Gra przeznaczona jest dla 2-4 graczy. Mogą nimi być również zawarte w grze boty. Celem jest zapisanie się i zaliczenie przedmiotów w czterech zakładach lub doprowadzenie pozostałych graczy do utraty wszystkich punktów.
\vspace{10pt}

%\paragraph*{2.Rozgrywka}
\noindent \textbf{2.Rozgrywka}\\
\noindent Każdy gracz porusza się po planszy zgodnie z ruchem wskazówek zegara. To, o ile gracz może się poruszyć w danej rundzie jest ustalane poprzez rzut dwiema kostkami. Jeśli wypadnie nam tzw.dublet, czyli takie same wartości na obu kostkach np. 5 i 5, wtedy idziemy na wskazane przez kostki pole i rzucamy ponownie. Po wyrzuceniu dubletu trzy razy pod rząd w tej samej turze, idziemy na pole "naprawa komputera". Po stanięciu na danym polu gracz wykonuje czynności związane z jego specyfiką, określoną w punkcie 4.
\vspace{10pt}

%\paragraph*{3.Waluta}
\noindent \textbf{3.Waluta}\\
\noindent Każdy gracz dostaje na starcie 300 punktów ECTS.
\vspace{10pt}

%\paragraph*{4.Plansza}
\noindent \textbf{4.Plansza}\\
Plansza składa się z 36 pól. Wyróżniamy wśród nich:\\
\noindent \textbf{a) START} - tutaj zaczyna się rozgrywka. Po każdym przejściu przez to pole, dostajemy dodatkowe 50 ECTS.\\
\noindent \textbf{b) zakłady} - jest ich 8. Każdy zakład składa się z dwóch lub trzech pól przedmiotów, oznaczonych tym samym kolorem.\\
\noindent \textbf{c) pola przedmiotów} - sa 22 takie pola, oznaczone kolorowym paskiem. Każde takie pole reprezentuje jeden z wykładanych w naszym instytycuie przedmiotów. Możemy zapisać się na niego poprzez oddanie punktów ECTS w ilości przypisanej do danego pola (oczywiście jeśli mamy na to środki). Aby zaliczyć dany przedmiot, musimy napisać z niego odpowiegnio kartkówkę, kolokwium i egzamin - każda taka operacja kosztuje pewną przypisaną ilość ECTS. Prace pisemne można pisać jedynie na przedmiocie, na który jesteśmy zapisani. Przedmiot uznajemy za zaliczony, jeśli napisaliśmy z niego egzamin. Uwaga - za jednym razem można wykonać tylko jedną z tych czynności. Jeśli staniemy na przedmiocie, na który zapisał się już ktoś inny, oddajemy tej osobie ilość ECTS przypisaną do danego pola. Możemy również zastąpić aktualnego posiadacza pola - kosztuje to tyle, ile wynosi aktualna wartość opla, powiększona o 100\%. Uwaga - nowy zapisany na przedmiot zaczyna go od początku - musi sam od początku napisać wszystkie prace pisemne.\\ 
\noindent \textbf{d) małe grzybki} - jest jedno takie pole. Stanięcie na nim skutkuje utratą 15 punktów ECTS.\\
\noindent \textbf{e) duże grzyby} - jest jedno takie pole. Stanięcie na nim skutkuje utratą 30. punktów ECTS.\\
\noindent \textbf{f) szansa} - jest 6 takich pól, oznaczonych [...?]. Po stanięcu na nim losuje się karta, która wskazuje, jaka akcja się wykona. Uwaga - nie ma możliwości uniknięcia wylosowanej czynności.\\
\noindent \textbf{g) tramwaj} - pole umożliwiające przejście do wybranego przez siebie innego pola. Uwaga - możemy iść tylko na posiadane przez nas lub wolne pole przedmiotu. \\
\noindent \textbf{h) zepsułeś komputer} - pole, które przenosi nas do pola naprawa komputera\\
\noindent \textbf{i) naprawa komputera} - jeśli staniemy na tym polu, musimy pozostać na nim przez 3 tury, zanim będzimy mogli je opuścić. Innym sposobem na opuszczenie tego pola jest oddanie 50 ECTS lub wylosowanie na dwóch kostkach dubletu.\\
\noindent \textbf{j) praktyki} - są 3 takie pola, Aby się na nie zapisać, trzeba oddać przypisaną ilość punktów ECTS. Praktyk nie można przejąć od kogoś - kto pierwszy, ten lepszy.\\ 
\vspace{10pt}

%\paragraph*{5.Co, jeśli staniemy na polu i nie stać na na opłacenie akcji związanej z nim?}
\noindent \textbf{5.Co, jeśli staniemy na polu i nie stać na na opłacenie akcji związanej z nim?}\\
\noindent W takiej sytuacji będziemy musieli wypisać się z co najmniej jednego przedmiotu - tak, aby było stać przynajmniej na opłacenie akcji. Jeśli nie jesteśmy zapisani na żaden przedmiot, przegrywamy i kończymy rozgrywkę. Uwaga - możemy to zrobić jedynie w sytuacji, w której jest to niezbędne do kontynuowania rozgrywki. Nie można tego zrobić na własne życzenie, np. w celu przejęcia zajętego przedmiotu. 
\vspace{10pt}

\noindent Dobór nazw przedmiotów i ich pozycji na planszy jest losowy, niekierowany naszymi preferencjami.
\end{document}


